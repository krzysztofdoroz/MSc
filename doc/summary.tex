\chapter{Summary}

\section{Main goals}

One of the main goal of this thesis was to find efficient ways of tackling the portfolio optimization problem.
Apart from co-evolutionary multi-agent system, many other approaches have been taken into consideration.
The essential part of work at the beginning consisted of gathering information about any feasible algorithm that could be adjusted to solve the problem. 
The main source of knowledge were books and papers devoted to the identical or very similar problem. 

The second goal was to implement the most promising algorithms found during first stage.
Four genuine approaches have been implemented: genetic algorithm, trend following, co-evolutionary and co-evolutionary multi-agent system.
The last two of mentioned algorithms were run on multi-agent platform. 

Last but not least, all implemented algorithms were supposed to be tested. 
Three sets of tests have been performed in order to assess the usefulness of each of them.


\section{Results}

As mentioned before, four algorithms have been implemented.
Trend following has been implemented as R script.
In case of the other three, Java was used.
Some are run on a multi-agent running platform implemented in Java.
Its architecture seems to be very simple but it provides all functionality that is needed.
Apart from that, to provide user friendliness, a group of shell scripts was created.
They build all necessary artifacts and deploy them.
They are further discussed in appendix A.
   
Three sets of tests have been performed to assess each algorithm performance during substantially different overall situation on a market.
Obtained results shown that it is difficult to pinpoint one approach that outperforms all others.
It pretty clear that each of them has some pros and cons.
Results are discusses in more detail in chapter \ref{cha:tests}.


\section{Conclusions}

All implemented algorithms have been exhaustively tested using historical data reflecting completely different situations on the market.
Taking into consideration both good and bad times leads to conclusions that all approaches have advantages as well as disadvantages.
That raises a question what algorithm to use if we want to build our own portfolio.
Trend following is good for unstable times because it somehow guarantees that the losses would be minimalised according to our preferences.
However, during stock market rise it is outperformed by other algorithms.
Clearly we would like to have means to highlight good aspects of each algorithm.  
Right now it is a little bit inconvenient therefore new system is proposed in \ref{sec:future}.
That system should expose all strengths of implemented approaches and be user friendly for a casual investor. 


\section{Future work}
\label{sec:future}

All tests have been performed in a way that mimicks real trading on a stock exchange.
Historical stock data have been used, but it is fairly obvious that the systems could be changed to actually invest money in a real world.
Right now the biggest obstacles to efficiently use them are:

\begin{itemize}
  \item feeding data to algorithms is a little bit cumbersome
  \item each algorithm is implemented as an independent entity
  \item systems are not user-friendly - running them requires using shell scripts
  \item modification of algorithms' parameters is inconvenient
  \item interpreting results is time consuming and could be enhanced
\end{itemize}
 
New system should combat all the problems mentioned above.
It should incorporate all implemented algorithms and use them as strategies.
Obviously investors should be able to modify parameters of running strategies via user-friendly GUI.
They could set them manually or they could choose from a set of predefined sets of parameters.
User should check what each of them is suggesting and make a decision about a portfolio composition.
Apart from that, access to charts comparing all proposed strategies would be provided.
That should give further insight into situation on the stock market and ease decision making.

All stock data should be downloaded during the start of application from Yahoo! (via YQL - Yahoo Query Language).
After downloading it would be stored in an underlying database.
In that case algorithms could fetch all the data they need (historical as well as most recent one) from one, central database.
 
