\chapter{Multi-Objective Optimization}
\label{cha:multiObjectiveOptimization}



%---------------------------------------------------------------------------

\section{Multi-Objective Optimization}
\label{sec:strukturaDokumentu}

Multi-objective optimization is used to solve problems involving more than one objective function.
Such problems are often encountered in real world so efficient methods of solving them have wide applications.
(\cite{Deb:2001:MOU:559152})

%TODO: MCDM, MOOP

Each conflicting objective corresponds to a different optimal solution.  
It is obvious that solution that is optimal with respect to one objective requires a compromise in other objectives.
Because of this reason, in order to compare two solutions we have to introduce ordering relation.

Pareto front is the set of choices that are Pareto efficient. 

%---------------------------------------------------------------------------

\section{Trend following}
\label{sec:trendFollowing}

%---------------------------------------------------------------------------

\section{Genetic Algorithms}
\label{sec:genAlgorithms}

Many computational problems require searching through a huge number of possibilities for solutions.
Such search problems can often benefit from an effective use of parallelism, in which many different possibilities are explored simultaneously in an efficient way. 
It turns out that the process of biological evolution could provide the inspiration for addressing these problems.
(\cite{Mitchell01})

\subsection{Elements of genetic algorithms}

Almost all genetic algorithms have some elements in common \cite{Mitchell01}:
\begin{itemize}
  \item populations of chromosomes
  \item selection according to fitness
  \item crossover to produce new offspring
  \item random mutation of new offspring
  \item fitness function
\end{itemize}

The GA processes populations of chromosomes, succesively replacing one such population with another (by using GA operators: mutation, selection, crossover, etc.). 
Each chromosome contains potential solution to problem (e.g. represented as binary string). 
Fitness functions are used to assess how well specific chromosome solves the problem.
The better fitting chromosomes are more likely to be involved in reproduction process.
That mimics the biological process of evolution - fitness of an organism is obtained by means of simple variation (mutation, recombination, etc.) and natural selection
thus propagating genetic material to future generations.
These rules seems to be simple but they are in fact responsible for extraordinary variety and complexity of the biosphere.
\cite{Mitchell01} 
 

 

\subsection{Genetic algorithms operators}

Following \cite{Mitchell01} - the simplest form of genetic algorithm involves three types of operators: selection, crossover and mutation. 

\begin{description}

\item[selection]
  This operator selects chromosomes in the population for reproduction.
  The fitter the chromosome, the more times it is likely to be selected to reproduce.
  
\item[crossover]
  This operator randomly chooses a locus and exchanges the subsequences before and after that locus between two chromosomes to create two offspring.
  For example, the strings 10000100 and 11111111 could be crossed over after the third locus in each to produce the two offspring 10011111 and 11100100.
  The crossover operator roughly mimics biological recombination between two single-chromosome organisms.

\item[mutation]
  This operator randomly flips some of the bits in a chromosome.
  For example, the string 00000100 might be mutated in its second position to yield 01000100.
  Mutation can occur at each bit position in a string with some probability, usually very small (e.g., 0.001).


\end{description}

\subsection{Pseudocode for Genetic Algorithm}


%---------------------------------------------------------------------------

\section{Evolutionary Algorithms}
\label{sec:evolAlgorithms}

Many computational problems require a computer program to be adaptive—to continue to perform well in a
changing environment.

