\chapter{Test Results}
\label{cha:pierwszyDokument}

\section{First Set of Tests}

First set of tests uses historical data from year 2009 (data of 3 different stocks have been used). 
Portfolio contained 3 assets (they are one of the biggest companies present on WSE):

\begin{itemize} 
  \item KGHM
  \item TPSA
  \item PKO
\end{itemize}

Multi-agent platform has been used to run algorithms.
It was configured in the following way: 

\begin{itemize}
  \item configured to simulate trading strategy throughout the entire year 2009
  \item all investing decisions are solely based on results from algorithms
  \item migration mechanism between Computing Nodes has been enabled
  \item two Computing Nodes (with appropriate algorithms) have been used to obtain results
\end{itemize}

Trend following algorithm has been tested without multi-agent running platform as it is a standalone R script.

\subsection{Trend Following}

\subsubsection{Short-term trend results}

As described in \ref{trend_following_impl} and \ref{sec:trendFollowing}, we can adjust our trading rules to seek out short-term trends
 (by changing the values of $N$ and $M$ in the \emph{Simple Moving Average} method).
With values $N = 10$, $M = 20$ the SMA method focused on short-term trends.
 

\begin{figure}[ht]
  
  \begin{center}
    \includegraphics[scale=.4]{rplot0.png}
  \end{center}
  \caption{Chart showing value of our portfolio in time}
  \label{fig:trend-short}
\end{figure}

There are easily visible periods of time (figure~\ref{fig:trend-short}) when the portfolio value is not changing. 
It is a time when our portfolio does not contain any assets (but we still have money). 
After a while, conditions change and trend following algorithm decides to buy some assets.


\subsubsection{Intermediate-term trend results}

In this particular test (with values: $N = 20$, $M = 40$), trend following algorithm has been adjusted to focus on intermediate-term trends.
 
\begin{figure}[H]
  \begin{center}
    \includegraphics[scale=.4]{rplot.png}
  \end{center}
  \caption{Chart showing value of our portfolio in time}
  \label{fig:trend-int}
\end{figure}

Results are presented in figure~\ref{fig:trend-int}, as can be seen they are slightly better compared with short-term trends approach.  

\subsubsection{Conclusions}

It turned out that focusing on short-term or intermediate-term trends leads to almost the same results.
Intermediate-term trend approach offers slightly better return from the investment.
In both cases we encountered periods of time when portfolio contained no assets because situation on the market was not good enough to invest in any of the available stocks.


%---------------------------------------------------------------------------

\subsection{Genetic Algorithm}

Results are presented in figure ~\ref{fig:gen-algo}.

\begin{figure}[ht]
  \begin{center}
    \includegraphics[scale=.5]{simple-genetic-algo.png}
  \end{center}
  \caption{Chart showing value of our portfolio in time}
  \label{fig:gen-algo}
\end{figure}

\subsection{Co-evolutionary algorithm}

Results presented in figure~\ref{fig:co_eval_return} clearly show that the return of our investment is much higher than any other algorithm could achieve.  
Analysing figure~\ref{fig:co_eval_risk} explains why the results are so good - the risk associated with our portfolio is substantially higher compared to other methods.
In this case algorithm was tuned to treat non-dominated solutions from return oriented subpopulation - this explains why the results provide so much return and risk. 

\begin{figure}[ht]
  \begin{center}
    \includegraphics[scale=.5]{co-evol-value-oriented.png}
  \end{center}
  \caption{Chart showing value of our portfolio in time}
  \label{fig:co_eval_return}
\end{figure}


\begin{figure}[ht]
  \begin{center}
    \includegraphics[scale=.5]{co-evol-risk-value-oriented.png}
  \end{center}
  \caption{Chart showing value of risk in time}
  \label{fig:co_eval_risk}
\end{figure}

\subsection{CoEMAS}

Figure~\ref{fig:agent_return} presents portfolio value in time while  figure~\ref{fig:agent_risk} presents the risk associated with our investments.

\begin{figure}[ht]
  \begin{center}
    \includegraphics[scale=.5]{agent-return.png}
  \end{center}
  \caption{Chart showing value of our portfolio in time (CoEMAS)}
  \label{fig:agent_return}
\end{figure}


\begin{figure}[ht]
  \begin{center}
    \includegraphics[scale=.5]{agent-risk.png}
  \end{center}
  \caption{Chart showing value of risk in time (CoEMAS)}
  \label{fig:agent_risk}
\end{figure}

\subsection{Conclusions}

First set of tests showed that all algorithms give reasonable results.
It is very hard to pinpoint one approach that is much better than others.
Co-evolutionary system (\ref{sec:co-evol-sys}) outperformed other algorithms in terms of return.
Such good results can be explained by the fact that trading strategy proposed by this algorithm is much riskier than using CoEMAS.
  
On the other hand, CoEMAS offers substantially lower return but is much more safer.
As figure~\ref{fig:agent_risk} shows, the risky moves are mixed with safe ones resulting in much more balanced strategy.

Trend following as well as genetic algorithm can not give us any information about the risk we are taking by investing according to results they provide.
On the other hand, we can specify rules (in the trend following approach) that should reflect our attitude to risk taking (\ref{sec:trend_following_fundamentals}).
With genetic algorithm (GA) we have no such option - as described in \ref{sec:gen_fitness_fun}, our ability to customize GA is limited.
Obviously we could test different values of coefficients responsible for simulating evolution process but it turns out that the fitness function that does not take risk into
account is the most serious limitation.

