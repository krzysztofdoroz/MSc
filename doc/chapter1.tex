\chapter{Introduction}
\label{cha:introduction}

The portfolio optimization problem is of paramount importance to each investor willing to risk their money in order to get potential benefits exceeding interest rates.
Before 1950s, people relied on common sense, experience or even premonition to construct their portfolios.
Then the scientists were able to formulate theories establishing the relation between risk and potential return of the investment.
Finally, investors had solid tools at hand to ease the uneasy process of investing money.
Obviously proposed theories will not make each of us a millionaire.
They can merely be used as a yet another source of analytical information that can be taken into consideration.

My thesis focuses on apllication of co-evolutionary multi-agent algorithms to solve the portfolio optimization problem.
Evolutionary algorithms are a large group of methods, which are mimicking natural process of evolution, solving optimization problems.
In order to even closer reflect the natural populations of living creatures, we basically introduce independent entities simulating interactions and behaviour of real creatures
competing for resources in a natural habitat.
That level of flexibility can be obtained by using multi-agent approach.
There are many other nature inspired algorithms like simulated annealing, ant colony optimization, etc.
They are all probabilistic metaheuristics so we have no guarantee that obtained solution is the optimal one.
In case of search problem with huge solution spaces, checking all feasible solutions would be impossible or would take too much time so the above heuristics are
 still useful.

   
Finally, some tests have been performed to verify whether the proposed approach would perform well confronted with historical data. 
     


%---------------------------------------------------------------------------

\section{Main goals}
\label{sec:mainGoals}

The main goal of this thesis is to find and apply efficient methods of solving the portfolio optimization problem. That task can be further split into three main steps.

First of all, it is necessary to find some robust methods of solving multi-objective optimization problems that would be appropriate to the problem mentioned in the thesis topic.

Secondly, all approaches selected to be feasible in the first step have to be adjusted to actually solve the portfolio optimization problem.
Most of the selected algorithms will give only a vague idea of how to deal with a general optimization problem so it will be necessary to actually
 implement and adapt them to this
 particular problem.

Finally, the tests based on historical data will be performed to assess the performance of each approach.

%---------------------------------------------------------------------------

\section{Content overview}
\label{sec:zawartoscPracy}

Detailed description of multi-objective optimization and portfolio optimization problem are covered in chapter \ref{cha:multiObjectiveOptimization}.
Apart from that, Capital Asset Pricing Model (CAPM), which is further used as a basis for fitness functions formulation.
To fully understand formulas associated with CAPM, some definitions of statistical functions are provided.
Then theoretical description of all algorithms that have been implemented is discussed.
That involves: trend following basics, evolutionary algorithms with particular emphasis on genetic algorithms, co-evolutionary approach and finally
the co-evolutionary multi-agent system is described in great details.
The ideas behind each of the algorithms are presented.



Chapter \ref{cha:implementedAlgorithms} discusses all implementation details of used algorithms.
The description is especially focuses on modifications of general approaches described in chapter \ref{cha:multiObjectiveOptimization}
 to solve the portfolio optimization problem.
Pseudocodes and UML class diagrams are provided to gain further insight into internals of each method.
After that, all used technologies are briefly described.

Information regarding technical details of running and modifying implemented algorithms are provided in appendix \ref{cha:user_guide}.
Each script is described in great length.

Test results are presented in chapter \ref{cha:tests}. 
Implemented algorithms have been tested on historical data so their performance can be compared with each other.
Three sets of tests have been performed and each of them represented completely different situation on the market or different system configuration.

Finally, a concept of brand new system using all implemented algorithms is briefly mentioned in \ref{sec:future}.
It would highlight all advantages of tested approaches and provide user friendly interface for investors using it.
They would not have to deal with cumbersome data feeding and interpreting results as that would be completely transparent to them.
Stock data would be downloaded from Yahoo service and kept in system's database.
 










