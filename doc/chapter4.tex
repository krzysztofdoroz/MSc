\chapter{Implemented Algorithms}
\label{cha:implementedAlgorithms}



%---------------------------------------------------------------------------

\section{Trend following}
\label{trend_following_impl}

The crux of the trend following method is to choose a set of entry and exit point conditions that determine our trading strategy.



\textbf{Entry points:}
  \begin{itemize}
    \item Simple moving average (SMA) of last $N$ days is greater than SMA of last $M$ days
    \item Current stock price is max of last $N$ days
  \end{itemize}

As soon as at least one of the above conditions is satisfied we go long.


\textbf{Exit points:}
  \begin{itemize}
    \item Losses on a single trade are greater than 2 \% 
    \item Simple moving average (SMA) of last $N$ days is lesser than SMA of last $M$ days
  \end{itemize}

As soon as the exit condition is satisfied we go short.
 

\subsection{Pseudocode}

\begin{description}

\item[SMA(i,N)]
  calculates Simple Moving Average for stock $i$, $N$ last days are taken into account  
\item[go\_short(i)]
  sell stock $i$
\item[go\_long(i)]
  buy stock $i$
\item[get\_current\_stock\_price(i, day)]
  returns stock $i$ price for specific $day$ 
\item[get\_most\_recent\_trade\_price(i)]
  returns the price we paid for stock $i$ (we have stock $i$ in our portfolio)
\item[max(i,N)]
  returns the maximum price for stock $i$ in the last $N$ days
\item[N, M, maximal\_value\_loss]
  parameters 
\end{description}
% 


\begin{algorithmic}

\STATE $maximal\_value\_loss \gets 0.98$

\FOR{$day = 1$ to $max\_day$} 

  \FOR{$i = 1$ to $number\_of\_stocks$}

    \IF {$ get\_most\_recent\_trade\_price(i) < maximal\_value\_loss * get\_current\_stock\_price(i, day) $} 
	    \STATE $go\_short(i)$
    \ELSE
	    \IF {$SMA(i,N) < SMA(i,M)$}
		    \STATE $go\_short(i)$
	    \ENDIF
    \ENDIF

    \IF {$SMA(i,N) > SMA(i,M)$} 
	    \STATE $go\_long(i)$
    \ELSE
	    \IF {$max(i,N)) < get\_current\_stock\_price(i, day)$}
		    \STATE $go\_long(i)$
	    \ENDIF
    \ENDIF

  \ENDFOR

\ENDFOR

\end{algorithmic}



\section{Genetic Algorithm}
\label{sec:genAlgoImpl}



%---------------------------------------------------------------------------

\section{Implementation Details}
\label{sec:implDetails}

\subsection{Precomputation of statistical functions}
\label{precompute}
Precomputation of statistical functions (like covariance, standard deviation etc.) from historical data is done by R scripts. 
Results are stored in files, later used as an input to other algorithms.

\subsection{Data Source component}

Data Source component aggregates all stock historical data as well as all precomputed statistical functions results from \ref{precompute}.   
Its interface enables access to all data it contains. Data Source is injected as a bean to other components by Spring.
They use it mainly as a source of data required by various algorithms.

\subsection{Computing Algorithm component}



\subsection{Multi-agent system architecture}

\newpage
\begin{figure}[!ht]
  \begin{center}
    \includegraphics[scale=.4]{agent_framework.png}
  \end{center}
  \caption{Deployment diagram of two Computing Nodes and Aggregation Node (in real situations more Computing Nodes are present)}
\end{figure}

Data Source as well as Computing Algorithm are created and injected to Computing Node(CN) by Spring. 
CNs send results of computations to Aggregating Node by JMS. 
Apart from that, JMS is used to provide migration capability to each CN.
The sole purpose of Aggregating Node is to gather results from CNs, choose the best non-dominated solution and write the results to file. 
Based on this file all charts are created and comparison of different algorithms is possible.

  

\subsection{Technologies used}

Trend following algorithm has been implemented as a R script.






